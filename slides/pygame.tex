% !TeX spellcheck = en_US
% !TeX program = xelatex

%Possible values for aspectratio are: 1610, 149, 54, 43 and 32.
\documentclass[aspectratio=169]{beamer}

% Use this package on Windows if you have any XeTeX/LuaTeX related compilation problem 
%\usepackage{fixltx2e}
\usepackage{lmodern}
\usepackage[T1]{fontenc}
\usepackage[utf8]{inputenc}
\usepackage{graphicx}
\usepackage[export]{adjustbox}
\usepackage{calc}
%\usepackage[obeyspaces,spaces]{url}

% Where to look for the pictures
% Define the folder where you will keep pictures for this particular presentation
% It is more convenient than precise a whole path for each picture independently
% Example:
\graphicspath{{./images/}}

% To configure where to look for the template sources
% This is the definition and some useful code to use for usefolder function
% You don't need to modify anything here. 
\makeatletter
  \def\beamer@calltheme#1#2#3{%
    \def\beamer@themelist{#2}
    \@for\beamer@themename:=\beamer@themelist\do
    {\usepackage[{#1}]{\beamer@themelocation/#3\beamer@themename}}}

  \def\usefolder#1{
    \def\beamer@themelocation{#1}
  }
  \def\beamer@themelocation{}

% Where actually to look for the template sources
% Put a path to the folder or, if you want, copy the corresponding sources into beamer themes' location of your latex distribution
% Example:
\usefolder{./sources/}

%% The name of the template to use
\usetheme{NokiaTheme}

% To use the blue version of the theme
%\UseBlueColor

% To show grid on the background
%\ShowGrid

% To define the information class
% By default it's "Nokia Internal Use"
\renewcommand{\Classifier}{Public}

% To change the year in copyright part
% Default is current year
%\renewcommand{\PresentationYear}{2018}

%%%%%%%%%%%
%
% My macros
%
%%%%%%%%%%%

\renewcommand{\emph}[1]{\textcolor{ClearTitle}{\bf #1}}
\newcommand{\memph}[1]{\textcolor{myred}{#1}}


%--------------------------------------------------------------------------------
%
\title{Introduction to Pygame}
\author{\normalsize François Durand\\[5pt]}
\date{\vspace{-.5cm}Python Workshop, 17 June 2020}

%--------------------------------------------------------------------------------
\begin{document}

% Title page is compiled using the title, author etc filled above
\begin{frame}
\titlepage
\end{frame}

\begin{frame}{Sources}
\url{https://fr.wikibooks.org/wiki/Pygame}, especially:
\begin{itemize}
\item \url{https://fr.wikibooks.org/wiki/Pygame/Introduction_\%C3\%A0_Pygame}
\item \url{https://fr.wikibooks.org/wiki/Pygame/Introduction_au_module_Sprite}
\item \url{https://fr.wikibooks.org/wiki/Pygame/Chimp_-_Ligne_par_ligne}
\end{itemize}
\end{frame}

\begin{frame}{What is Pygame?}
\begin{itemize}
\item \emph{Pygame} is a Python package that wraps the SDL library.
\item \emph{SDL (Simple DirectMedia Layer)} is a cross-platform development library written in C, designed to provide access to \emph{multimedia}: 2D graphics, audio, keyboard, mouse and joystick.
\item Pygame can be used to design \emph{games} but not only: anything that makes uses of a graphical interface, sound, etc. And in particular, it can be used for \emph{demos}.
\item Pygame is \emph{not} designed to make \emph{GUIs} (with buttons, menus, etc). For that, you can use Tkinter.
\end{itemize}
\end{frame}

\begin{frame}[fragile]{Minimal Example}
Cf.  \verb|minimal_example.py|.
\begin{itemize}
\item \verb|pygame.init()|: initialize all \emph{modules} of Pygame, such as \verb|pygame.display|, \verb|pygame.event|, \verb|pygame.image|, \verb|pygame.time|, \verb|pygame.sprite|, \verb|pygame.mixer|,  \verb|pygame.key|, \verb|pygame.mouse|, \verb|pygame.font| and \verb|pygame.transform|.
\item \verb|screen = pygame.display.set_mode(size)|: create a \emph{Surface} object with the image to display.
\item \verb|pygame.event.get()|: catches \emph{events} (quit, key strokes, etc).
\end{itemize}
\end{frame}

\begin{frame}[fragile]{Basics: Bouncing Ball (1)}
Cf.  \verb|bouncing_ball.py|.

\begin{itemize}
\item \verb|clock = pygame.time.Clock()|: launch a \emph{clock}.
\item \verb|ball = pygame.image.load("ball.gif").convert_alpha()|: create a \emph{Surface} object with the image of the ball.
\begin{itemize}
\item The \verb|convert_alpha()| is not strictly necessary, but it is a good habit to have.
\item Depending on whether you want to deal with transparency, use \verb|convert()| or \verb|convert_alpha()|.
\end{itemize} 
\item \verb|ball_rect = ball.get_rect()|: create a \emph{Rect} object representing a rectangular zone.
\begin{itemize}
\item A Rect is just a convenient way to store \verb|x|, \verb|y|, \verb|width|, \verb|height|.
\item But it also gives access to attributes (\verb|left|, \verb|right|, \verb|top|, \verb|bottom|, \verb|center|, \verb|topleft|...) and methods (\verb|move|, \verb|colliderect|...).
\item Default initialization: \verb|x = y = 0|.
\end{itemize}
\end{itemize}
\end{frame}

\begin{frame}[fragile]{Basics: Bouncing Ball (2)}
Cf.  \verb|bouncing_ball.py|.

\begin{itemize}
\item \verb|ball_rect.move_ip(ball_speed)|: modify \verb|ball_rect| ``in place'' so that it contains the \emph{new coordinates}.
\begin{itemize}
\item Variant: \verb|ball_rect = ball_rect.move(ball_speed)|.
\end{itemize}
\item \verb|screen.fill(black)|: to start preparing the new version of the image, we cover everything with a black background, but the new version of the image is not displayed yet (\emph{double buffer}).
\item \verb|screen.blit(ball, ball_rect)|: \emph{copy} the pixels of \verb|ball| on \verb|screen|, in the coordinates given by \verb|ball_rect|.
\item \verb|pygame.display.update()|: actually \emph{display} the prepared image.
\begin{itemize}
\item Variant: \verb|pygame.display.flip()| would do the same thing, but has less options.
\end{itemize}
\item \verb|clock.tick(100)|: \emph{wait} 0.01 s.
\end{itemize}
\end{frame}


\begin{frame}[fragile]{More on Surface and Rect: Bouncing Ball in a Box}
Cf.  \verb|bouncing_ball_in_a_box.py|.
\begin{itemize}
\item \verb|box_empty|: a Surface to store the picture of the \emph{empty box}.
\item \verb|box|: a Surface where we draw \emph{the box + the ball}.
\item \verb|ball_rect|: note that \verb|ball_rect|, by itself, does not ``know'' that its coordinates are relative to the box. They are just \emph{plain old numbers}.
\end{itemize}
\end{frame}

\begin{frame}[fragile]{Sprite: Bouncing Balls}
Cf.  \verb|bouncing_balls.py|.
\begin{itemize}
\item \verb|class Ball(pygame.sprite.Sprite)|: a \emph{Sprite} must have attributes \verb|image| and \verb|rect|, and a method \verb|update|.
\item \verb|def update(self)|: the method \emph{need not deal with display}, it just says what the sprite should do when an update is asked, typically once per frame. 
\item \verb|all_sprites = pygame.sprite.RenderPlain([Ball() for _ in| \verb|range(10)])|: generally, a Sprite should belong to one or more \emph{Group} objects. There are several subclasses of Group with various capabilities: GroupSingle, RenderPlain, RenderClear, RenderUpdates. For most usages, RenderPlain is fine.
\item \verb|all_sprites.update()|: call the method \verb|update| of all sprites.
\item \verb|all_sprites.draw(screen)|: call the built-in method \verb|draw| of all sprites.
\end{itemize}
\end{frame}

\begin{frame}[fragile]{Remark: Back to The Single Bouncing Ball Example}
\verb|bouncing_ball_sprite.py|.

This new version using Sprite is slightly longer than the original, but it is \emph{cleaner} and therefore easier to \emph{maintain} and \emph{expand}.
\end{frame}

\begin{frame}[fragile]{Collision and Sound: Bouncing Balls with Collision}
Cf. \verb|bouncing_balls_collision.py|
\begin{itemize}
\item \verb|self.explode_sound = pygame.mixer.Sound('balloon_pop.wav')|: create a \emph{Sound} object.
\item \verb|self.explode_sound.play()|: \emph{play} the sound.
\item \verb|pygame.sprite.spritecollide(red_ball, yellow_balls, dokill=1)|: return a \emph{list} of the yellow balls collided by the red ball.
\begin{itemize}
\item \verb|dokill=1|: remove the collided balls from the group.
\item By default, collision is determined only by the \verb|rect| attribute of each sprite, but it is possible to specify a custom function for collision detection (next slide).
\end{itemize}
\end{itemize}
\end{frame}

\begin{frame}[fragile]{Cleaner Collision: Bouncing Balls with Collision, revisited}
Cf. \verb|bouncing_balls_collision_circle.py|
\begin{itemize}
\item \verb|self.radius = self.rect.width / 2|: our sprite will need a \verb|radius| attribute.
\item \verb|pygame.sprite.spritecollide(red_ball, yellow_balls, dokill=1,| \verb|collided=pygame.sprite.collide_circle)|: use the method \verb|collide_circle| for collision detection.
\begin{itemize}
\item Other methods are available, in particular \verb|collide_rect_ratio|. Used with a ratio $< 1$, it gives often satisfactory results without needing to compute an exact ``hit mask''.
\end{itemize}
\end{itemize}
\end{frame}

\begin{frame}[fragile]{Group Collision: Bouncing Balls with Contamination}
Cf. \verb|bouncing_balls_contamination.py|.
\begin{itemize}
\item \verb|pygame.sprite.groupcollide(healthy\_balls, contaminated\_balls,| \verb|0, 0)|: \emph{list} the healthy balls that collide with contaminated balls.
\item \verb|healthy_balls.remove(ball)|: \emph{remove} the ball from the group.
\item \verb|contaminated_balls.add(ball)|: \emph{add} the ball to the group.
\end{itemize}
\end{frame}

\begin{frame}[fragile]{Use the Keyboard: Collide Them All!}
Cf. \verb|collide_them_all.py|.
\begin{itemize}
\item \verb|pressed_keys = pygame.key.get_pressed()|: a \emph{tuple of Booleans}, indicating for each key if it is pressed or not.
\item \verb|pressed_keys[pygame.K_UP]|: the constant \verb|pygame.K_UP| is the \emph{number} representing the ``up arrow'' key.
\item In \verb|BallPlayer.update|, the syntax with multiple ``ifs'' accounts for the cases where several keys are pressed \emph{simultaneously}.
\end{itemize}
\end{frame}

\begin{frame}[fragile]{A Simple Game: Chimp (1)}
Cf. \verb|chimp.py|.
\begin{itemize}
\item \verb|if not pygame.font|: when a module's import fails, its value is \emph{None}.
\item \verb|def load_image|, \verb|def load_sound|: it is convenient to have \emph{import functions} that are a bit more sophisticated than the built-in ones.
\item \verb|image.set_colorkey(colorkey, pygame.RLEACCEL)|: use a color in the original image as representing \emph{transparency}.
\item \verb|pos = pygame.mouse.get_pos()|: get the position of the \emph{mouse}.
\item \verb|hit_box = self.rect.inflate(-5, -5)|: define a \emph{slightly smaller} rect (to be used for collision detection).
\end{itemize}
\end{frame}

\begin{frame}[fragile]{A Simple Game: Chimp (2)}
Cf. \verb|chimp.py|.
\begin{itemize}
\item \verb|self.image = pygame.transform.flip(self.image, 1, 0)|: \emph{flip the image} of the chimp when changing direction.
\item \verb|self.image = pygame.transform.rotate(self.original_image,| \verb|self.rotation_angle)|: \emph{rotate the image} of the chimp when hit.
\item \verb|pygame.display.set_caption("Monkey Fever")|: change the \emph{title} of the window.
\item \verb|pygame.font|: this module is used to put \emph{text} on a Surface.
\end{itemize}
\end{frame}

\begin{frame}[fragile]{Summary: Pygame Modules}

\begin{itemize}
\item \verb|pygame.display|: the display window.
\item \verb|pygame.event|: catch events like \verb|pygame.QUIT|, \verb|pygame.KEYDOWN|, etc.
\item \verb|pygame.image|: images.
\item \verb|pygame.time|: clock.
\item \verb|pygame.sprite|: sprites.
\item \verb|pygame.mixer|: sounds.
\item \verb|pygame.key|: keyboard.
\item \verb|pygame.mouse|: mouse.
\item \verb|pygame.font|: text.
\item \verb|pygame.transform|: transform images.
\item There are several others, such as \verb|pygame.draw| to draw simple shapes like lines, polygons, etc.
\end{itemize}
\end{frame}

\begin{frame}{Thanks For Your Attention!}
\begin{center}	
\includegraphics[width=14cm]{image_conclu03}
\end{center}	
\end{frame}

% "Nokia" final slide
\begin{frame}
\end{frame}

\end{document}
